\section*{Задание 1}
Написать хвостовую рекурсивную функцию my-reverse, которая развернет верхний
уровень своего списка-аргумента lst.

\subsection*{Решение}
\begin{lstinputlisting}[label=third,caption=Решение задания №1, language=lisp, firstline=2, lastline=7]{../lab.lisp}
\end{lstinputlisting}

\section*{Задание №2}
Написать функцию, которая возвращает первый элемент списка-аргумента, который сам
является непустым списком.

\subsection*{Решение}
\begin{lstinputlisting}[label=third,caption=Решение задания №2, language=lisp, firstline=10, lastline=13]{../lab.lisp}
\end{lstinputlisting}

\section*{Задание №3}
Написать функцию, которая выбирает из заданного списка только те числа, которые
больше 1 и меньше 10.

(Вариант: между двумя заданными границами. )
\subsection*{Решение}
\begin{lstinputlisting}[label=third,caption=Решение задания №3, language=lisp, firstline=16, lastline=19]{../lab.lisp}
\end{lstinputlisting}

\section*{Задание №4}
Напишите рекурсивную функцию, которая умножает на заданное число-аргумент все
числа
из заданного списка-аргумента.

\subsection*{Решение}
Все элементы списка --- числа.
\begin{lstinputlisting}[label=third,caption=Решение задания №4.1, language=lisp, firstline=22, lastline=24]{../lab.lisp}
\end{lstinputlisting}

Элементы списка -- любые объекты.
\begin{lstinputlisting}[label=third,caption=Решение задания №4.2, language=lisp, firstline=26, lastline=30]{../lab.lisp}
\end{lstinputlisting}

\section*{Задание №5}
Напишите функцию, select-between, которая из списка-аргумента, содержащего только
числа, выбирает только те, которые расположены между двумя указанными границами-
аргументами и возвращает их в виде списка (упорядоченного по возрастанию списка чисел
(+ 2 балла)).

\subsection*{Решение}
\begin{lstinputlisting}[label=third,caption=Решение задания №5, language=lisp, firstline=33, lastline=39]{../lab.lisp}
\end{lstinputlisting}


\section*{Задание №6}
Написать рекурсивную версию (с именем rec-add) вычисления суммы чисел заданного
списка.
\subsection*{Решение}

Одноуровневый смешанный список.
\begin{lstinputlisting}[label=third,caption=Решение задания №6.1, language=lisp, firstline=42, lastline=47]{../lab.lisp}
\end{lstinputlisting}

Структурированный список.

\begin{lstinputlisting}[label=third,caption=Решение задания №6.2, language=lisp, firstline=49, lastline=55]{../lab.lisp}
\end{lstinputlisting}

\section*{Задание №7}
Написать рекурсивную версию с именем recnth функции nth.
\begin{lstinputlisting}[label=third,caption=Решение задания №7, language=lisp, firstline=58, lastline=61]{../lab.lisp}
\end{lstinputlisting}

\section*{Задание №8}
Написать рекурсивную функцию allodd, которая возвращает t когда все элементы списка
нечетные.
\subsection*{Решение}
\begin{lstinputlisting}[label=third,caption=Решение задания №8, language=lisp, firstline=64, lastline=69]{../lab.lisp}
\end{lstinputlisting}	

\section*{Задание №9}
Написать рекурсивную функцию, которая возвращает первое нечетное число из списка
(структурированного), возможно создавая некоторые вспомогательные функции.
\subsection*{Решение}
\begin{lstinputlisting}[label=third,caption=Решение задания №8, language=lisp, firstline=72, lastline=76]{../lab.lisp}
\end{lstinputlisting}

\section*{Задание №10}
Используя cons-дополняемую рекурсию с одним тестом завершения,
написать функцию которая получает как аргумент список чисел, а возвращает список
квадратов этих чисел в том же порядке
\subsection*{Решение}
\begin{lstinputlisting}[label=third,caption=Решение задания №8, language=lisp, firstline=79, lastline=81]{../lab.lisp}
\end{lstinputlisting}


