\section*{Задание 1}
\subsection*{Постановка задачи}
Написать функцию, которая по своему аргументу-списку \texttt{lst} определяет, является ли он полиндромом (то есть равны ли \texttt{lst} и \texttt{(reverse lst)})

\subsection*{Решение}
\begin{lstinputlisting}[label=third,caption=Решение задания №1, language=lisp, firstline=2, lastline=6]{../lab.lisp}
	
\end{lstinputlisting}


\section*{Задание №2}
\subsection*{Постановка задачи}
Написать предикат \texttt{set-equal}, который возвращает \texttt{t}, если два его множества-аргумента содержат одни и те же элементы, порядок которых не имеет значения

\subsection*{Решение}
\begin{lstinputlisting}[label=third,caption=Решение задания №2, language=lisp, firstline=12, lastline=19]{../lab.lisp}
	
\end{lstinputlisting}


\section*{Задание №3}
\subsection*{Постановка задачи}
Напишите необходимые функции, которые обрабатывают таблицу из точечных пар: \texttt{(страна . столица)}, и возвращают по стране столицу, а по столице --- страну

\subsection*{Решение}
\begin{lstinputlisting}[label=third,caption=Решение задания №3, language=lisp, firstline=25, lastline=33]{../lab.lisp}
\end{lstinputlisting}

\section*{Задание №4}
\subsection*{Постановка задачи}
Напишите функцию \texttt{swap-first-last}, которая переставляет в списке аргументе первый и последний элементы

\subsection*{Решение}
\begin{lstinputlisting}[label=third,caption=Решение задания №4, language=lisp, firstline=39, lastline=42]{../lab.lisp}
\end{lstinputlisting}

\section*{Задание №5}
\subsection*{Постановка задачи}
Напишите функцию \texttt{swap-two-ellement}, которая переставляет в списке-аргументе два указанных своими порядковыми номерами элемента в этом списке

\subsection*{Решение}
\begin{lstinputlisting}[label=third,caption=Решение задания №5, language=lisp, firstline=47, lastline=57]{../lab.lisp}
\end{lstinputlisting}

\section*{Задание №6}
\subsection*{Постановка задачи}
Напишите две функции, \texttt{swap-to-left} и \texttt{swap-to-right}, которые производят круговую перестановку в списке-аргументе влево и вправо, соответственно

\subsection*{Решение}
\begin{lstinputlisting}[label=third,caption=Решение задания №6, language=lisp, firstline=62, lastline=69]{../lab.lisp}
\end{lstinputlisting}

\section*{Задание №7}
\subsection*{Постановка задачи}
Напишите функцию, которая добавляет к множеству двухэлементных списков новый двухэлементный список, если его там нет.

\subsection*{Решение}
\begin{lstinputlisting}[label=third,caption=Решение задания №7, language=lisp, firstline=75, lastline=86]{../lab.lisp}
\end{lstinputlisting}

\section*{Задание №8}
\subsection*{Постановка задачи}
Напишите функцию, которая умножает на заданное число-аргумент первый числовой
элемент списка из заданного 3-х элементного списка-аргумента, когда\\

a) все элементы списка -- числа,

6) элементы списка -- любые объекты

\subsection*{Решение}
\begin{lstinputlisting}[label=third,caption=Решение задания №8, language=lisp, firstline=92, lastline=98]{../lab.lisp}
\end{lstinputlisting}

\section*{Задание №9}
\subsection*{Постановка задачи}
Напишите функцию, select-between, которая из списка-аргумента из 5 чисел выбирает
только те, которые расположены между двумя указанными границами-аргументами и
возвращает их в виде списка.

\subsection*{Решение}
\begin{lstinputlisting}[label=third,caption=Решение задания №9, language=lisp, firstline=104, lastline=110]{../lab.lisp}
\end{lstinputlisting}

\section*{Контрольные вопросы}
\textbf{Вопрос 1.} Структуроразрушающие и не разрушающие структуру списка функции.

\textbf{Ответ.}
Функции, реализующие опреации со списками, делятся на две группы:
\begin{enumerate}
	\item не разрушающие структуру функции; данные функции не меняют переданный им объект-аргумент, а создают копию, с которой в дальнейшем производят необходимые преобразования; к таким функциям относятся: append, reverse, last, nth, nthcdr, length, remove, subst и др.
	\item структуроразрушающие функции; данные функции меняют сам объект-аргумент, из-за чего теряется возможность работать с исходным списком; чаще всего имя структуроразрушающих функций начинается с префикса -n: nreverse, nconc, nsubst и др.
\end{enumerate}

Обычно в Lisp существуют функции-дубли, которые реализуют одно и то же преобразование, но по разному (с сохранением структуры и без): apppend/nconc, reverse/nreverse и т.д.\newline

\textbf{Вопрос 2.} Отличие в работе функций cons, list, append, nconc и в их результате.

\textbf{Ответ.}
Функция \textbf{cons} - чисто математическая, она принимает ровно 2 аргумента, создает бианрный узел и расставляет указатели (car - на первый аргумент, cdr - на второй). В результате работы функции может получится как точечная пара, так и список (зависит от второго аргумента).\\

Функция \textbf{list} - это форма, она принимает произвольное количество аргументов и создает из них список. В отличии от функции cons, list создает столько бинарных узлов, сколько передано ей аргументов, и связывает их вместе. Результатом работы данной функции всегда будет список.\\

Функция \textbf{append} также является формой. Она принимает на вход произвольное число аргументов. Для всех аргументов, кроме последнего, эта функция создает копию, ссылая при этом последний элемент каждого списка аргумента на первый элемент следующего по порядку списка аргумента. В результате работы функции append может получится как список, так и точечная пара (зависит от последнего аргумента). \\

Итого: \textbf{cons} создает однин бинарный узел, \textbf{list} создает столько бинарных узлов, сколько передано аргументов, \textbf{append} cоздает копии всех бинарных узлов для каждого из аргументов, исключая последний аргумент.\\