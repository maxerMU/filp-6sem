\section*{Задание 1}
Представить следующие списки в виде списочных ячеек:
	\begin{enumerate}
		\item ’(open close halph)
		
		\item ’((open1) (close2) (halph3))
		
		\item ’((one) for all (and (me (for you))))
		
		\item ’((TOOL)(call))
		
		\item ’((TOOL1)((call2))((sell)))
		
		\item ’(((TOOL)(call))(sell))
	\end{enumerate}
	
	\textbf{Решение}
	Решение оформлено на тетрадном листе бумаге, прилагающемуся к отчету.	
	
\section*{Задание 2}

Используя только функции CAR и CDR, написать выражения, возвращающие второй, третий, четвертый элементы заданного списка.
	
\textbf{Решение}

\begin{lstlisting}[label=first,caption=Решение задания №2, language=lisp]
(car (cdr '(0 1 2 3)))
(car (cdr (cdr '(0 1 2 3))))
(car (cdr (cdr (cdr '(0 1 2 3)))))
\end{lstlisting}

\section*{Задание №3}

Что будет в результате вычисления выражений?

\textbf{Решение}

\begin{lstlisting}[label=second,caption=Решение задания №3, language=lisp]
(caadr '((blue cube) (red pyaramid))) 
; ((red pyramid)) => (red pyramid) => red
(cdar '((abc) (def) (ghi)))
; (abc) => Nil
(cadr '((abc) (def) (ghi))) 
; ((def) (ghi)) => def
(caddr '((abc) (def) (ghi)))
; ((def) (ghi)) => ((ghi)) => (ghi)
\end{lstlisting}

\section*{Задание 4}
Напишите результат вычисления выражений:

\textbf{Решение}

\begin{lstlisting}[label=second,caption=Решение задания №4, language=lisp]
(list 'Fred 'and  'Wilma) ; (Fred and Wilma)
(cons 'Fred '(and Wilma)) ; (Fred and Wilma)
(list 'Fred '(and Wilma)) ; (Fred (and Wilma))
(cons 'Fred '(Wilma)) ; (Fred Wilma)
(cons Nil Nil) ; (Nil)
(list Nil Nil) ; (Nil Nil)
(cons T Nil) ; (T)
(list T Nil) ; (T Nil)
(cons Nil T) ; (Nil . T)
(list Nil T) ; (Nil T)
(list Nil) ; (Nil)
(cons T (list Nil)) ; (T Nil)
(cons '(T) Nil) ; ((T))
(list '(T) Nil) ; ((T) Nil)
(list '(one two) '(free temp)) ; ((one two) (free temp))
(cons '(one two) '(free temp)) ; ((one two) free temp)
\end{lstlisting}

\section*{Задание №5}

\indent Написать функцию (f ar1 ar2 ar3 ar4), возвращаущую список: ((ar1 ar) (ar3 ar4)).""\newline
\indent Написать функцию (f ar1 ar2), возвращаущую ((ar1) (ar2)).""\newline
\indent Написать функцию (f ar1), возвращаущую (((ar1))).""\newline
\indent Представить результаты в виде списочных ячеек. 
\textbf{Решение}
\begin{lstlisting}[label=third,caption=Решение задания №5 (функция №1), language=lisp]
(DEFUN f (ar1 ar2 ar3 ar4) (list (list ar1 ar2) (list ar3 ar4) ))
\end{lstlisting}
Списочные ячейки функции 1 представлены на рисунке \ref{img:1.png}
\imgs{1.png}{h!}{0.5}{Cписочные ячейки функция 1}

\begin{lstlisting}[label=third,caption=Решение задания №5 (функция №2), language=lisp]
(defun f (ar1 ar2) (list (list ar1) (list ar2)) )
\end{lstlisting}
Списочные ячейки функции 2 представлены на рисунке \ref{img:2.png}
\imgs{2.png}{h!}{0.5}{Cписочные ячейки функция 2}

\begin{lstlisting}[label=third,caption=Решение задания №5 (функция №3), language=lisp]
(defun f (ar1) (list (list (list ar1))) )
\end{lstlisting}
Списочные ячейки функции 2 представлены на рисунке \ref{img:3.png}
\imgs{3.png}{h!}{0.5}{Cписочные ячейки функция 3}

\section*{Контрольные вопросы}
	\textbf{Вопрос 1.} Элементы языка: определение, синтаксис, представление в памяти.
	
	Элементами языка Lisp являются атомы и структуры (точечные пары, списки). К атомам относятся:
	\begin{itemize}
		\item символы -- набор литер, начинающихся с буквы.
		\item специальные символы: $\{T, Nil\}$ (используются для обозначения логических констант).
		\item самоопределимые атомы -- натуральные, дробные, вещественные числа, строки (последовательность символов, заключенных в двойные апострофы)
	\end{itemize}
	
		Точечные пары ::= (<атом>, <атом>) | (<атом>, <точечная пара>) |
	
	(<точечная пара>, <атом>) | (<точечная пара>, <точечная пара>)
	
	""\newline
	\indent Список ::= <пустой список> | <непустой список>, где
	
	<пустой список> ::= () | Nil, <непустой список> ::= (<первый элемент>, <хвост>),
	
	<первый элемент> ::= <S-выражение>, <хвост> ::= <список>""\newline
	
	\indent \textbf{Список} -- частный случай S-выражения. Любая структура (точечная пара или список) заключаются в круглые скобки:
	
\begin{itemize}
	\item (A . B) -- точечная пара;
	\item $(A)$ -- список из одного элемента;
	\item $Nil$ или $()$ -- пустой список;
	\item (A . (B . (C . (D ()))))) или (A B C D) -- непустой список;
	\item Элементы списка могут являться списками: $((A)(B)(CD))$
\end{itemize}
	
Любая непустая структура в Lisp, в памяти представленна списковой ячейкой, хранящей два указателя: на голову и хвост.

	
\textbf{Вопрос 2.} Особенности языка Lisp. Структура программы. Символ апостроф.

Важной особенностью языка Lisp является единая синтаксическая форма записи программ и данных, что позволяет обрабатывать структуры данных как программы и модифицировать программы как данные. 

Lisp-программа представляет собой последовательность вычислимых выражений, явяляющихся атомом или списком.

Символ ' эквивалентен функции quote – он блокирует вычисление выражения. Таким образом, выражение воспринимается интерпретатором как данные.""\newline

\textbf{Вопрос 3.} Базис языка Lisp. Ядро языка. \newline

Базис языка образуют атомы, структуры (точечные пары и списки), базовые функции, базовые функционалы (функции, аргументами и значением которых являются функции).

Ядро Лиспа работает следующим образом: 
\begin{enumerate}
\item ожидает ввода S-выражения; 
\item передает введенное S-выражение функции EVAL (которая вычисляет значение своего единственного аргумента и возвращает его в качестве результата); 
\item выводит полученный результат;
\item переходит к пункту 1.
\end{enumerate}