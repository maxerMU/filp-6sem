\section*{Задание 1}
\subsection*{Постановка задачи}

Составить диаграмму вычисления следующих выражений:

\begin{enumerate}
	\item (equal 3 (abs -3))
	\item (equal (+ 1 2) 3)
	\item (equal (* 4 7) 21)
	\item (equal (* 2 3) (+ 7 2))
	\item (equal (- 7 3) (* 3 2)))
	\item (equal (abs (- 2 4)) 3)
\end{enumerate}

\subsection*{Решение}
Решение оформленно на тетрадном листке бумаге. К отчету прилагается.

\section*{Задание 2}
\subsection*{Постановка задачи}

Написать функцию, вычисляющую гипотенузу прямоугольного треугольника по заданным катетам и составить диаграмму ее вычисления. Решение.

\subsection*{Решение}

\begin{lstlisting}[label=second,caption=Решение задания №2, language=lisp]
(defun hypotenuse (a b) (sqrt (+ (* a a) (* b b))))
\end{lstlisting}

\section*{Задание 3}
\subsection*{Постановка задачи}
Написать функцию, вычисляющую объем параллелепипеда по 3-м его сторонам, и составить диаграмму ее вычисления.

\subsection*{Решение}

\begin{lstlisting}[label=third,caption=Решение задания №3, language=lisp]
(defun volume (a b c) (* a b c))
\end{lstlisting}

Диаграмма оформлена на тетрадном листе, прикрепленном к отчету.

\section*{Задание 4}
\subsection*{Постановка задачи}
Каковы результаты вычисления следующих выражений? (объяснить возможную ошибку и
варианты ее устранен6ия)

\subsection*{Решение}

\begin{lstlisting}[label=4xd,caption=Решение задания №4, language=lisp]
(list 'a c) -> The variable C is unbound.
;(list 'a 'c)
(cons 'a 'b 'c) -> invalid number of arguments: 3
;(cons 'a (cons 'b 'c))
(cons 'a (b c)) -> Undefined function: B Undefined variable: C
;(cons 'a (cons 'b 'c))
(list 'a (b c)) -> Undefined function: B Undefined variable: C
;(list 'a 'b 'c)
(cons 'a '(b c)) -> (A B C)
(list a '(b c)) ->  The variable A is unbound.
(caddy (1 2 3 4 5)) -> Undefined function CADDY
;(caddr '(1 2 3 4 5))
(list (+ 1 '(length '(1 2 3)))) -> The value (LENGTH '(1 2 3)) is not of type NUMBER
;(list (+ 1 (length '(1 2 3))))

\end{lstlisting}

\section*{Задание 5}
\subsection*{Постановка задачи}
Написать функцию \textbf{longer\_then} от двух списков- аргументов, которая возвращает T, если первый аргумент имеет большую длину.

\subsection*{Решение}

\begin{lstlisting}[label=5xd,caption=Решение задания №5, language=lisp]
(defun longer_than (a b) (> (length a) (length b)))
\end{lstlisting}

\section*{Задание 6}
\subsection*{Постановка задачи}
Каковы результаты вычисления следующих выражений? 

\subsection*{Решение}

\begin{lstlisting}[label=6xd,caption=Решение задания №6, language=lisp]
(cons 3 (list 5 6)) -> (3 5 6)
(cons 3 `(list 5 6)) -> (3 LIST 5 6)
(list 3 `from 8 `gives (- 9 3))) -> 3 FROM 9 GIVES 6
(+ (length for 2 too)) (car '(21 22 23))) -> The variable FOR is unbound
; (+ (length '(for 2 too)) (car '(21 22 23))) -> 24
(cdr `(cons is short for ans)) -> (IS SHORT FOR ANS)
(car (list one two)); VARIABLE ONE IS UNBOUND
(car (list `one `two)) -> ONE

\end{lstlisting}

\section*{Задание 7}
\subsection*{Постановка задачи}
Дана функция \textbf{(defun mystery (x) (list (second x) (first x)))}. Какие результаты вычисления следующих выражений?

\subsection*{Решение}

\begin{lstlisting}[label=7xd,caption=Решение задания №7, language=lisp]
(mystery (one two)) -> The variable TWO is unbound.
(mystery one 'two)) -> The variable ONE is unbound.
(mystery (last one two)) -> The variable ONE is unbound.
(mystery free) -> The variable FREE is unbound.

\end{lstlisting}

\section*{Задание 8}
\subsection*{Постановка задачи}
Написать функцию, которая переводит температуру в системе Фаренгейта
температуру по Цельсию (defum f-to-c (temp)...)

\subsection*{Решение}

\begin{lstlisting}[label=7xd,caption=Решение задания №8, language=lisp]
(defun f-to-c (temp) (* (/ 5 9) (- temp 320)))
(f-to-c 451) -> 655 / 9
\end{lstlisting}

\section*{Задание 9}
\subsection*{Постановка задачи}
Что получится при вычисления каждого из выражений?

\subsection*{Решение}

\begin{lstlisting}[label=7xd,caption=Решение задания №8, language=lisp]
(list 'cons t NIL) -> (cons T NIL)
(eval (list 'cons t NIL)) -> (T)
(eval (eval (list 'cons t NIL))) -> Undefined function: T
(apply #cons "(t NIL))
(eval NIL) -> Nil
(list 'eval NIL) -> (eval NIL)
(eval (list 'eval NIL)) -> Nil
\end{lstlisting}



\section*{Контрольные вопросы}

\textbf{Вопрос 1.} Базис Lisp. \newline
\indent\textbf{Ответ. }
Базис языка представлен:
\begin{itemize}
	\item структурами и атомами;
	\item функциями;
\end{itemize}

Функции, входящие в базис языка:
\begin{itemize}
	\item atom, eq, cons, car, cdr;
	\item cond, quote, lambda, eval, label.
\end{itemize}


\textbf{Вопрос 2.} Классификация функций языка Lisp.

\textbf{Ответ.} 

\begin{itemize}
	\item чистые (с фиксированным количеством аргументов) математические функции;
	\item рекурсивные функции;
	\item специальные функции – формы (принимают произвольное количество аргументов или по разному обрабатывают аргументы);
	\item псевдофункции (создающие «эффект» – отображающие на экране процесс обработки данных и т.п.);
	\item функции с вариативными значениями, выбирающие одно значение;
	\item функции высших порядков – функционалы (используются для построения синтаксически управляемых программ);
\end{itemize}

\textbf{Вопрос 3.} Способы создание функций.

\textbf{Ответ.}
\begin{itemize}
	\item lambda выражения (lambda $\lambda$-список форма)
	\item defun (defun f $\lambda$-выражение)
\end{itemize}


\textbf{Вопрос 4.} Функции \textbf{car}, \textbf{cdr}.

\textbf{Ответ.} Функции $car$, $cdr$ являются базовыми функциями доступа к
данным. car принимает точечную пару или список в качестве аргумента
и возвращает первый элемент или $Nil$, соответственно. $cdr$ принимает точечную пару или список в качестве аргумента и возвращает все элементы
кроме первого или $Nil$, соответственно.""\newline

\textbf{Вопрос 5.} Функции \textbf{list}, \textbf{cons}.

\textbf{Ответ.} Функции $list$, $cons$ являются функциями создания списков
($cons$ – базовая, $list$ – нет). $cons$ создает списочную ячейку и устанавливает два указателя на аргументы. $list$ принимает переменное число аргументов и возвращает список, элементы которого – переданные в функцию
аргументы.